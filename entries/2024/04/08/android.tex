% % % % mainfile: ../../../../master.tex
\subsection{Meeting with iSprint}
\label{task:20240408_android}

\subsubsection{Minutes}

Attendees: Prof. Shar, Albert, Vikas, Phong Phan, Yan Naing, Wei Minn

1. Albert's agenda: 1) Publication of research on this discovered malware, and 2) Discuss plans for the next iteration of Shieldcon this year.

2. Albert would like to continue holding Shieldcon with SMU; he is eager to find a better venue than the previous time; the exact timing of the conference depends on the finding of the malware; he plans to leverage publicity of the conference. 

3. Albert mentioned the preliminary survey conducted on the banks regarding measures for the vulnerabilities in their apps. Many banks did not respond to this malware attack.

4. Phong presents FjordPhantom, virtualization-based malware. He demonstrated a container can run on any phone and requires no root no root required. It utilizes virtualization to run a container that wraps a malware that hooks the target app that looks identical to bank app

5. This type of malware is very flexible and module; can easily repackage any android app.

6. Albert Chings said they discoverd another similar malware a few weeks ago; but still conducting experiments to confirm.

7. This malware relys on sideloading of APK into your Android phone.

8. Phong's Demo of FjordPhantom on emulator:

a. Install demo bank APK via sideloading

b. Install screenreader app via sideloading

c. Demo bank app warns of enabled accessibility, and do not allow you to proceed

d. Inject container inside the app using a Python script and re-sideload the now tampered OCBC app

e. With modified OCBC app, even with screenreader and accessibility turned on, there is no warning message, and the user can proceed.

9. The container also contains Hooking framework which plays essential role in this atack.

10. Albert asserted that more than 80\% of the bank apps are susceptible to this attack

11. Albert blames this vulnerability on poorly written apps.

12. Albert calls for more research and warn the public about this malware.

13. Prof. Shar aims for ASE conference which will be around July.

14. Vikas asked asked about the company that first detected this malware at the end of 23, and inquired about the differences between their release and i-Sprint's plans; Albert confirmed that they are partners with the company that discovered it.

15: Albert said he is planning an extension to last year's paper, and asked Prof. Shar of any better way to proceed.

16: Prof. Shar plans on focusing on this particular attack, rather than as a continuation of last year's work.

17: Albert suggested two separate papers: 1) this attack, and 2) pregress of previous work one year later.

18: Prof. Shar insists to focus on attack for the time being.

19: Albert shares about the concerns that CSA and other agencies across the region have about this attack; he also talked about the potential for more publicity for i-Sprint and SMU regarding the release of this malware.

20: Prof. Shar asked Albert on how much information can be disclose in the paper with regards to the banks, and the detials of the attacks?

21: Albert replies that we cannot divulge the affected banks' names, and instead can follow the press releases of i-Sprint. He added that the banks have already been warned.

22: Prof. Shar assigns Wei Minn to take the lead for this paper.

23. After brief discussion about the Shieldcon date, we agreed to meet up again in person for more details like last time.

24: Prof. Shar asked for 1 or 2 weeks to get back to i-Sprint.

25: Albert, Phong and Vikas leaves Meeting

26: Prof. Shar tasked Wei Minn with setting the skeleton for the paper (follow Vikas' example from last time) and get David involved only toward the end.
