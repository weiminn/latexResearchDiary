 % mainfile: ../../../../master.tex
\label{task:20240402_aosp}

\subsection{Complete ART Execution Flow}

\subsubsection{How is Java Code loaded into Android Runtime?}

Deep Dive into ART\footnote{\url{https://www.youtube.com/watch?v=mFq0vNvUgj8}}, and Method Execution Flow\footnote{\url{https://huanle19891345.github.io/en/android/art/jni/java_jni\%E6\%96\%B9\%E6\%B3\%95\%E8\%B0\%83\%E7\%94\%A8\%E5\%8E\%9F\%E7\%90\%86/}}, LinkCode\footnote{\url{https://cloud.tencent.com/developer/article/2343052}}\footnote{\url{https://blog.csdn.net/luoshengyang/article/details/39533503}}, Loading OAT files\footnote{\url{https://blog.csdn.net/luoshengyang/article/details/39307813}}, Executing Methods\footnote{\url{https://blog.csdn.net/Luoshengyang/article/details/40289405}}, Virtual Machine\footnote{\url{https://blog.csdn.net/luoshengyang/article/details/8914953}}, Loading Classes and Methods\footnote{\url{https://blog.csdn.net/luoshengyang/article/details/39533503}}

\subsubsection{Loading OAT Files}

An Application has only one \texttt{classes.dex} file which is compiled into and OAT file via LLVM compiler and stored in the \texttt{oatexec} section of the OAT file. It is also possible to load DEX file during run time while are also translated into OAT.

\subsubsection{Loading OAT Methods}

\texttt{FindClass} in \path{jni_internal.cc} accesses singleton instance of the Runtime, singleton instance of \texttt{ClassLinker}, and \texttt{ClassLoader} instance. And the pointer to the \texttt{ClassLoader} instance is passed into the \texttt{FindClass} function of \path{class_linker.cc}.



% \includepdf[pages=-, scale=.95,pagecommand={}]{entries/2024/01/01/art.pdf}

% \begin{itemize}
% \item \textbf{Domain.} The context of the process that is acting upon something.
% \item \textbf{Type.} The context of the resource on which the process is acting.
% \item \textbf{Class.} The object class of the resource (e.g. \textit{file} or \textit{socket}).
% \item \textbf{Permissions.} The permissions that are allowed given the \textit{domain}, \textit{type} and \textit{class}.
% \end{itemize}

% SELinux rule syntax:


% \subsubsection{Decoding Permission Denial Message}

% Message:
% \begin{lstlisting}
% type=AVC msg=audit(1363289005.532:184): avc:  denied  { read } for  pid=29199 comm="Trace" 
% name="online" dev="sysfs" ino=30 scontext=staff_u:staff_r:googletalk_plugin_t 
% tcontext=system_u:object_r:sysfs_t tclass=file
% \end{lstlisting}

% \begin{longtable}{p{.15\linewidth}p{.15\linewidth}p{.65\linewidth}} 
% \toprule
% Log part & Name & Description \\
% \midrule
% \endhead

% \texttt{type=AVC}
% &Log type
% &Only in the \texttt{audit.log} file; it informs the user what kind of audit log type this is. 
% \\

% \texttt{msg=audit(1363289005.532:184)}
% &Timestamp
% &Timestamp in seconds since epoch, meaning the number of seconds since January 1st, 1970. You can convert this to a more human readable format using date -d @ followed by the number, like so: \texttt{date -d @1363292159.532}.
% \\

% \texttt{avc:}
% &Log type (again)
% &
% \\

% \texttt{ino=30}
% &inode number
% &The inode number of the target file. In this case, since we know it is on the \texttt{sysfs} file system, we can look for this file using: \texttt{find /sys -xdev -inum 30}
% \\

% \texttt{scontent=staff\_u:staff\_r:googletalk\_plugin\_t}
% &Source context
% &The security context of the process (the domain)
% \\

% \texttt{tcontext=system\_u:object\_r:sysfs\_t}
% &Target context
% &The security context of the target resource (in this case the file)
% \\

% \texttt{tclass=file}
% &Target class
% &The class of the target.
% \\

% \midrule
% \caption{Permission Denied Syntax} 
% \label{tab:permissiondeniedsyntax}
% \end{longtable}


% \subsubsection{SELinux Architecture}

% SELinux consists of four main components: object managers (OM), access vector cache (AVC), security server, and security policy as show below:
% \begin{figure}[H]
%     \centering
%     \includegraphics[width=.85\linewidth]{entries/2023/12/10/selinux.png}
%     \caption{SELinux Components}
%     \label{fig:selinux}
% \end{figure}
