\label{task:20240517_android}

\subsection{What makes up FjordPhantom exactly?}

\subsubsection{Virtual Container}

Allows installation of the same app multiple times to be used with different accounts\footnote{\url{https://promon.co/security-news/fjordphantom-android-malware/}}. 

Those different apps run in the same sandbox to allow them to access each other's files and memory to debug each other. Allows breaking of sandbox without rooting.

Loads additional code (logic of malware and invocations of hooking framework) into the process of hosted app.

\subsubsection{Repackaging}

Most popular code injection method is to repackage with additional code. But can be easily detected.

\subsubsection{Hooking}

Hook into APIs that is used to determine if Accessibility services are turned on, and return false information to bypass screenreader and rooting detection.

Can also hook to UI functionalities such as dialog boxes to close them immediately if they contain certain strings.

Usage of hooking in Malware analysis:

\begin{longtable}{p{.3\linewidth}p{.7\linewidth}} 
\toprule
Paper & Hook Logic  \\
\midrule
\endhead
Ronin, DASC2018\footnote{\url{https://www.researchgate.net/publication/328611051_Android_Hooking_Revisited}}
&Hook Main activity to call Ronin

Ronin overwrites PLT tables\footnote{\url{https://www.technovelty.org/linux/plt-and-got-the-key-to-code-sharing-and-dynamic-libraries.html}}  for hooking native functions in the memory
\\

Boxify, USENIX2015\footnote{\url{https://www.usenix.org/system/files/conference/usenixsecurity15/sec15-paper-backes.pdf}}
& Secure sandboxing by dynmaically loading and executing within its own process.

No firmware or application modification, or rooting needed.
\\

Malton, USENIX2017\footnote{\url{https://www.usenix.org/system/files/conference/usenixsecurity17/sec17-xue.pdf}}
& Non-invasive Multi-layer (Java, C++ and ARM) Taint Analysis and Path Exploration. 

Taint propagation at Instruction (ARM) layer using Valgrind \footnote{\url{https://valgrind.org/docs/valgrind2007.pdf}} because both app and framework code are compiled to native code in the ART runtime.
\\

ARTDroid, IMPS2016\footnote{\url{https://ceur-ws.org/Vol-1575/paper_10.pdf}}
&
\\

RetroSkeleton, MobiSys'13\footnote{\url{https://www.cs.ucdavis.edu/~hchen/paper/mobisys2013rs.pdf}}
& Hooking apps by static instrumentation
\\

AppGuard, TACAS2013\footnote{\url{https://link.springer.com/content/pdf/10.1007/978-3-642-36742-7_39.pdf}}
& Hooking apps by static instrumentation
\\

StaDynA, CODASPY'15\footnote{\url{https://lilicoding.github.io/SA3Repo/papers/2015_zhauniarovich2015stadyna.pdf}}
& Hooking framework APIs
\\

TaintDroid, USENIX2010\footnote{\url{https://static.usenix.org/event/osdi10/tech/full_papers/Enck.pdf}}
&Hooking taint sources such as sensor manager, information databases and device identifers.

Also hooking taint sinks (native socket library for network connections)
\\

DroidScope, USENIX2012\footnote{\url{https://www.usenix.org/system/files/conference/usenixsecurity12/sec12-final107.pdf}}
& Runtime and kernel instrumentation
\\

Andrubis, Technical Report\footnote{\url{https://www.vvdveen.com/publications/TR0414001.pdf}}
& Runtime instrumentation
\\

\midrule
\caption{Permission Denied Syntax} 
\label{tab:permissiondeniedsyntax}
\end{longtable}

\pagebreak