 % mainfile: ../../../../master.tex
\subsection{Figure out NatiDroid for AOSP14}
\label{task:20240214_aosp}

\subsubsection{Solved Java part of Natidroid}

\begin{lstlisting}[language=bash]
git clone https://github.com/Sable/soot.git
git checkout develop
sudo apt install maven
sudo update-alternatives --config java
mvn clean compile assembly:single
rm -rf out && mkdir out && javac src/*.java src/javaLink/*.java src/cppLink/*.java src/util/*.java -cp "./libs/*" -d out && java -cp "./out:./soot/target/sootclasses-trunk-jar-with-dependencies.jar" findJavaLink
\end{lstlisting}

Workspace:

\begin{lstlisting}[language=bash]
# terminal 1
rm -rf out && mkdir out && javac src/*.java src/javaLink/*.java src/cppLink/*.java src/util/*.java -cp "./soot/target/sootclasses-trunk-jar-with-dependencies.jar:./libs/fastjson-1.2.73.jar" -d out && java -cp "./out:./soot/target/sootclasses-trunk-jar-with-dependencies.jar" findJavaLink
# terminal 2
clear && find ./ \( -iname '*.java' -a ! -iname '*test*' \) | xargs grep --color -in 'internalTransform.*;'
# terminal 3
mvn clean compile assembly:single    
\end{lstlisting}


% \includepdf[pages=-, scale=.95,pagecommand={}]{entries/2024/01/01/art.pdf}

% \begin{itemize}
% \item \textbf{Domain.} The context of the process that is acting upon something.
% \item \textbf{Type.} The context of the resource on which the process is acting.
% \item \textbf{Class.} The object class of the resource (e.g. \textit{file} or \textit{socket}).
% \item \textbf{Permissions.} The permissions that are allowed given the \textit{domain}, \textit{type} and \textit{class}.
% \end{itemize}

% SELinux rule syntax:
% \begin{lstlisting}
% allow <domain> <type>:<class> { <permissions> };
% \end{lstlisting}

% \subsubsection{Decoding Permission Denial Message}

% Message:
% \begin{lstlisting}
% type=AVC msg=audit(1363289005.532:184): avc:  denied  { read } for  pid=29199 comm="Trace" 
% name="online" dev="sysfs" ino=30 scontext=staff_u:staff_r:googletalk_plugin_t 
% tcontext=system_u:object_r:sysfs_t tclass=file
% \end{lstlisting}

% \begin{longtable}{p{.15\linewidth}p{.15\linewidth}p{.65\linewidth}} 
% \toprule
% Log part & Name & Description \\
% \midrule
% \endhead

% \texttt{type=AVC}
% &Log type
% &Only in the \texttt{audit.log} file; it informs the user what kind of audit log type this is. 
% \\

% \texttt{msg=audit(1363289005.532:184)}
% &Timestamp
% &Timestamp in seconds since epoch, meaning the number of seconds since January 1st, 1970. You can convert this to a more human readable format using date -d @ followed by the number, like so: \texttt{date -d @1363292159.532}.
% \\

% \texttt{avc:}
% &Log type (again)
% &
% \\

% \texttt{ino=30}
% &inode number
% &The inode number of the target file. In this case, since we know it is on the \texttt{sysfs} file system, we can look for this file using: \texttt{find /sys -xdev -inum 30}
% \\

% \texttt{scontent=staff\_u:staff\_r:googletalk\_plugin\_t}
% &Source context
% &The security context of the process (the domain)
% \\

% \texttt{tcontext=system\_u:object\_r:sysfs\_t}
% &Target context
% &The security context of the target resource (in this case the file)
% \\

% \texttt{tclass=file}
% &Target class
% &The class of the target.
% \\

% \midrule
% \caption{Permission Denied Syntax} 
% \label{tab:permissiondeniedsyntax}
% \end{longtable}


% \subsubsection{SELinux Architecture}

% SELinux consists of four main components: object managers (OM), access vector cache (AVC), security server, and security policy as show below:
% \begin{figure}[H]
%     \centering
%     \includegraphics[width=.85\linewidth]{entries/2023/12/10/selinux.png}
%     \caption{SELinux Components}
%     \label{fig:selinux}
% \end{figure}
