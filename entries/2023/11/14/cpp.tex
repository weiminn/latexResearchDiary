% mainfile: ../../../../master.tex
\subsection{C++ Primer}
\label{task:20231114_cpp}

\subsubsection{Primitive Built-in Types}

Includes \textbf{arithmetic types} and a special type named \textbf{void} which has no associated values and can be used in only a few circumstances, most commonly as the return type for functions that do not return a value.

The arithmetic types are divided into two categories: \textbf{integral types} (which include character and boolean types) and floating-point types.

\begin{longtable}{p{.2\linewidth}  p{.40\linewidth} p{.3\linewidth}} 
\toprule
\textbf{Type} & \textbf{Meaning} & \textbf{Minimum Size}\\
\midrule

\texttt{bool} & boolean (\texttt{true} or \texttt{false}) & NA \\
\texttt{char} & character & 8 bits \\
\texttt{w\_char\_t} & wide character & 16 bits\\
\texttt{char16\_t} & Unicode character & 16 bits \\
\texttt{char32\_t} & Unicode character & 32 bits \\
\texttt{short} & short integer & 16 bits \\
\texttt{int} & integer & 16 bits \\
\texttt{long} & long integer & 32 bits \\
\texttt{long long} & long integer & 64 bits \\
\texttt{float} & single-precision floating-point & 6 significant digits \\
\texttt{double} & double-precision floating-point & 10 significant digits\\
\texttt{long double} & extended-precision floating-point & 10 significant digits \\

\midrule
\caption{Zygote Initialization Sequence} 
\label{tab:zygoteinitializationsequence}
\end{longtable}

Except for \texttt{bool} and extended character types, the integral types may be \textbf{signed} (can represent negative or positive numbers) or \textbf{unsigned}. By default, \texttt{int}, \texttt{short}, \texttt{long}, \texttt{long long} are all signed. To declare unsigned type, prepend \texttt{unsigned} to the type. \texttt{char} is signed on some machine and \texttt{unsigned} on others, and \texttt{unsigned int} is abbreviated as \texttt{unsigned}.

Conversions happen automatically when we use an object of one type where an object of another type is expected. 
\begin{lstlisting}[language=c++]
unsigned u = 10; 
int i = -42; 
std::cout << i + i << std::endl; // prints -84 
std::cout << u + i << std::endl; // if 32-bit ints, prints 4294967264
\end{lstlisting}
In the above snippet, converting a negative number to \texttt{unsigned} will cause the value to "wrap around" because \texttt{unsigned} values can never be less than $0$. Thus, extra care should be taken if we want to write loops with \texttt{unsigned} values and stopping conditions at negative values like the snippet below:
\begin{lstlisting}[language=c++]
// WRONG: u can never be less than 0; the condition will always succeed 
for (unsigned u = 10; u >= 0; --u) 
    std::cout << u << std::endl;
\end{lstlisting}

As such it is always advisable to not mix \texttt{signed} and \texttt{unsigned} types. By default, integer literals (\texttt{42}) are signed, while octal (\texttt{024}) and hexadecimal (\texttt{0x14}) may be signed or unsigned.

% Escape sequences are nonprintable characters with special meanings and begin with backlash:
% \begin{longtable}{p{.2\linewidth}  p{.40\linewidth} p{.3\linewidth}} 
% \toprule
% \textbf{Description} & \textbf{Escape Sequence} \\
% \midrule

% new line & \texttt{\\n} \\
% vertical tab & \texttt{\\v} \\
% horizontal tab & \texttt{\\t} \\
% backspace & \texttt{\\b} \\
% carriage return & \texttt{\\r} \\
% alert (bell) & \texttt{\a} \\
% single quote & \texttt{\\'} \\
% double quote & \texttt{\\"} \\
% formfeed & \texttt{\\f} \\
% backslash & \texttt{\\\\} \\

% \midrule
% \caption{Escape sequences} 
% \label{tab:escapesequences}
% \end{longtable}

Escape sequences are used as if they were single characters:
\begin{lstlisting}[language=c++]
std::cout << `\n`; // prints a newline 
std::cout << "\tHi!\n"; // prints a tab followd by "Hi!" and a newline
\end{lstlisting}

\subsubsection*{Variables}

Initialization is not assignment. Initialization happens when a variable is given a value when it is created. Assignment obliterates an object's current value and replaces that value with a new one. 

Four different ways to initialize:
\begin{lstlisting}[language=c++]
int units_sold = 0; 
int units_sold = {0}; // list initialization; does not work for built-in types if data loss is likely
int units_sold{0}; 
int units_sold(0);
\end{lstlisting}

Variables defined outside any function body are initialized to zero by default.  Variables of built-in type defined inside a function are \textbf{uninitialized} and therefore undefined. Objects of class type that we do not explicitly initialize have a value that is defined by the class.

A \textbf{declaration} makes a name known to the program. A file that wants to use a name defined elsewhere includes a declaration for that name. A \textbf{definition} creates the associated entity. A definition involves declaration, allocates storage and may provide the variable with an initial value. 
\begin{lstlisting}[language=c++]
extern int i;   // declares but not define j
int j;          // declares and defines j
int k = 12;     // declares, defines and initializes j
extern double pi = 3.14;     // definition
\end{lstlisting}

Variables must be defined only once but can be declared several times. To use a variable in more than one file requires declarations that are separate from the variable's definition. To use the same variable in multiple files, we must define that in one - and only one - file. Other files that use that variable must declara - but not define - that variable.