% % % mainfile: ../../../../master.tex
\subsection{C++ Data Structure}
\label{task:20231122_cpp}

\subsubsection{\texttt{Sales\_item} Struct}

The data structure does not support any operations\footnote{Basically, class with only attributes, and no methods.} any requires the user to implement the operations themselves:
\begin{lstlisting}[language=C++]
struct Sales_data { 
    std::string bookNo; 
    unsigned units_sold = 0; 
    double revenue = 0.0; 
};
\end{lstlisting}
The names defined inside the class must be unique within the class but can reuse names defined outside the class. We define data members the same way that we define normal variables: We specify a base type followed by a list of one or more declarators.

The close curly that ends the class body must be followed by a semicolon. The semicolon is needed because we can define variables after the class body:
\begin{lstlisting}[language=C++]
struct Sales_data { /* ... */ } accum, trans, *salesptr; 
// equivalent, but better way to define these objects 
struct Sales_data { /* ... */}; Sales_data accum, trans, *salesptr;
\end{lstlisting}
The semicolon marks the end of the (usually empty) list of declarators. Ordinarily, it is a bad idea to define an object as part of a class definition. Doing so obscures the code by combining the definitions of two different entities—the class and a variable—in a single statement.

We use the dot operator (\texttt{.}) to read into the member attributes of the object:
\begin{lstlisting}[language=C++]
Sales_data data1, data2;
double price = 0; // price per book, used to calculate total revenue 

// read the first transactions: ISBN, number of books sold, price per book 
std::cin >> data1.bookNo >> data1.units_sold >> price; 

// calculate total revenue from price and units_sold 
data1.revenue = data1.units_sold * price;

// read the second transaction 
std::cin >> data2.bookNo >> data2.units_sold >> price; 
data2.revenue = data2.units_sold * price;
\end{lstlisting}

\subsubsection{Preprocessor}

In order to ensure that the class definition is the same in each file, classes are usually defined in header files. Typically, classes are stored in headers whose name derives from the name of the class. Thus, we will define our \texttt{Sales\_data} class in a header file named \texttt{Sales\_data.h}.

Headers often need to use facilities from other headers. For example, because our \texttt{Sales\_data} class has a string member, \texttt{Sales\_data.h} must \texttt{\#include} the \texttt{string} header. As we’ve seen, programs that use \texttt{Sales\_data} also need to include the \texttt{string} header in order to use the \texttt{bookNo} member. As a result, programs that use \texttt{Sales\_data} will include the \texttt{string} header twice: once directly and once as a side effect of including \texttt{Sales\_data.h}. Because a header might be included more than once, we need to write our headers in a way that is safe even if the header is included multiple times.

The \textbf{preprocessor} is a program that runs before the compiler and changes the source text of our programs. Our programs already rely on one preprocessor facility, \texttt{\#include}. When the preprocessor sees a \texttt{\#include}, it replaces the \texttt{\#include} with the contents of the specified header.

Header guards can be defined using the preprocessor. Preprocessor variables have one of two states: defined and not defined. \texttt{\#define} directive takes a name as a preprocessor variable. \texttt{\#ifdef} is true if the variable has been defined, and \texttt{\#ifndef} is true if the variable has not been defined. If the test is true, then everything following the \texttt{\#ifdef} or \texttt{\#ifndef} is processed up to the matching \texttt{\#endif}:
\begin{lstlisting}[language=C++]
#ifndef SALES_DATA_H 

#define SALES_DATA_H 
#include <string> 

struct Sales_data { 
    std::string bookNo; 
    unsigned units_sold = 0; 
    double revenue = 0.0; 
};

#endif
\end{lstlisting}
The first time \texttt{Sales\_data.h} is included, the \texttt{\#ifndef} test will succeed. The preprocessor will process the lines following \texttt{\#ifndef} up to the \texttt{\#endif}. As a result, the preprocessor variable \texttt{SALES\_DATA\_H} will be defined and the contents of \texttt{Sales\_data.h} will be copied into our program. If we include \texttt{Sales\_data.h} later on in the same file, the \texttt{\#ifndef} directive will be false. The lines between it and the \texttt{\#endif} directive will be ignored.

Preprocessor variables, including names of header guards, must be unique throughout the program. Typically we ensure uniqueness by basing the guard’s name on the name of a class in the header. To avoid name clashes with other entities in our programs, preprocessor variables usually are written in all uppercase.