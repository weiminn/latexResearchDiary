% % % mainfile: ../../../../master.tex
\subsection{C++ Types and Data Structures}
\label{task:20231122_cpp}

\subsubsection{Aliases}

Traditionally, we use \textbf{\texttt{typedef}} for synonym for another type:
\begin{lstlisting}[language=C++]
typedef double wages; // wages is a synonym for double 
typedef wages base, *p; // base is a synonym for double, p for double*
\end{lstlisting}


The new standard introduced a second way to define \textbf{alias declaration} type alias:
\begin{lstlisting}[language=C++]
using SI = Sales_item; // SI is a synonym for Sales_item
\end{lstlisting}

It is also possible to declare "pointer to" alias:
\begin{lstlisting}[language=C++]
typedef char *pstring; 
const pstring cstr = 0; // equivalent to char *const cstr = 0; 
const pstring *ps; // equivalent to const char *ps;
\end{lstlisting}

\subsubsection{\texttt{auto} Type Specifier}

We can let the compiler deduce the type from the initializer for us by using the \texttt{auto} type specifier:
\begin{lstlisting}[language=C++]
// the type of item is deduced from the type of the result of adding val1 and val2 
auto item = val1 + val2; // item initialized to the result of val1+val2
\end{lstlisting}

The initializaer for all the variables muxt have types that are consistent with each other:
\begin{lstlisting}[language=C++]
auto i = 0, *p = &i; // ok: i is int and p is a pointer to int 
auto sz = 0, pi = 3.14; // error: inconsistent types for sz and pi
\end{lstlisting}

\texttt{auto} ignores top-level \texttt{const}s, and only low-level initializer are usually kept:
\begin{lstlisting}[language=C++]
auto i = 0, *p = &i; // ok: i is int and p is a pointer to int 
auto sz = 0, pi = 3.14; // error: inconsistent types for sz and pi
\end{lstlisting}

% When we have variables whose value we know cannot be changed, and We want to prevent code from inadvertently giving a new value to the variable, we define the variable's type as \texttt{const}: 
% \begin{lstlisting}[language=C++]
% const int bufSize = 512; // input buffer size
% bufSize = 512; // error: attempt to write to const object
% \end{lstlisting}

% Because we can't change the value of a const object after we create it, it must be initialized:
% \begin{lstlisting}[language=C++]
% const int i = get_size(); // ok: initialized at run time 
% const int j = 42; // ok: initialized at compile time 
% const int k; // error: k is uninitialized const
% \end{lstlisting}

% \subsubsection*{\texttt{const} is Local to the File}

% When a \texttt{const} object is initialized from a compile-time constant as in:
% \begin{lstlisting}[language=C++]
% const int bufSize = 512;
% \end{lstlisting}
% the compiler will usually replace uses of the variable with its corresponding value during compilation. Thus, when we split a program to multiple files, every file that uses that \texttt{const} must have a access to its initializer. In order to see the initializer, the variable must be defined in every file that wants to use the variable's value. To support this, we define the \texttt{const} in one file, and declare it in other files that use that object.
% \begin{lstlisting}[language=C++]
% // file_1.cc defines and initializes a const that is accessible to other files 
% extern const int bufSize = fcn(); 

% // file_1.h 
% extern const int bufSize; // same bufSize as defined in file_1.cc
% \end{lstlisting}
% Because \texttt{bufSize} is \texttt{const}, we must specify \texttt{extern} in order for \texttt{bufSize} to be used in other files. \texttt{extern} signifies that \texttt{bufSize} is not local to this file and that its definition will occur elsewhere.

% \subsubsection*{References to \texttt{const}}

% Unlike an ordinary reference, a reference to \texttt{const} cannot be used to change the object to which the reference is bound:
% \begin{lstlisting}[language=C++]
% const int ci = 1024; 
% const int &r1 = ci; // ok: both reference and underlying object are const 
% r1 = 42; // error: r1is a reference to const int &r2 = 
% ci; // error: nonconstreference to a constobject
% \end{lstlisting}

% We can bind a reference to a \texttt{const} to a non\texttt{const} object, literals, or a more general expression:
% \begin{lstlisting}[language=C++]
% int i = 42;
% const int &r1 = i; // we can bind a const int& to a plain int object 
% const int &r2 = 42; // ok: r1 is a reference to const 
% const int &r3 = r1 * 2; // ok: r3 is a reference to 
% const int &r4 = r * 2; // error: r4is a plain, nonconst reference
% \end{lstlisting}
% It is important to realize that a reference to \texttt{const} restricts only what we can do through that reference. Binding a reference to \texttt{const} to an object says nothing about whether the underlying object itself is \texttt{const}.

% \subsubsection{Pointers and \texttt{const}}

% Like a reference to \texttt{const}, a pointer to \texttt{const} may not be used to change the object to which the pointer points. We may store the address of a \texttt{const} object only in a pointer to \texttt{const}, where we can modify the pointer itself (change the reference stored inside) but not the object (value) pointed to:
% \begin{lstlisting}[language=C++]
% const double pi = 3.14; // pi is const; its value may not be changed 
% double *ptr = &pi; // error: ptr is a plain pointer 
% const double *cptr = &pi; // ok: cptr may point to a double that is const 
% *cptr = 42; // error: cannot assign to *cptr
% \end{lstlisting}
% Like a reference to \texttt{const}, a pointer to \texttt{const} says nothing about whether the object to which the pointer points is \texttt{const}, that there is no guarantee that an object pointed to by a pointer to \texttt{const} won't change.

% We can have a pointer that is itself \texttt{const}, and the address it holds cannot by changed. We indicate that the pointer is \texttt{const} by putting the \texttt{const} by the \texttt{const}:
% \begin{lstlisting}[language=C++]
% int errNumb = 0; 
% int *const curErr = &errNumb; // curErr will always point to errNumb 
% const double pi = 3.14159; 
% const double *const pip = &pi; // pip is a constpointer to a const object

% *pip = 2.72; // error: pip is a pointer to const  and cannot be changed as the pointer is const
% // if the object to which curErrpoints (i.e., errNumb) is nonzero 
% if (*curErr) { 
%     errorHandler(); 
%     *curErr = 0; // ok: reset the value of the object to which curErr is bound because errNumb is not const 
% }
% \end{lstlisting}

% \subsubsection*{Constant Expressions}

% A constant expression is an expression whose value cannot change and that can be evaluated at compile time. A literal is a constant expression. A \texttt{const} object that is initialized from a constant expression is also a constant expression:
% \begin{lstlisting}[language=C++]
% const int max_files = 20; // max_files is a constant expression 
% const int limit = max_files + 1; // limit is a constant expression 
% int staff_size = 27; // staff_size is not a constant expression 
% const int sz = get_size(); // sz is not a constant expression
% \end{lstlisting}
% Although \path{staff_size} is initialized from a literal, it is not a constant expression because it is a plain \path{int}, not a \path{const int}. Even though \path{sz} is a \path{const}, the value of its initializer is not known until run time, and thus, \path{sz} is not a constant expression.

% We can ask the compiler to verify that a variable is a constant expression by declaring the variable in a \texttt{constexpr} declaration. Variables declared as \path{constexpr} are implicitly \path{const} and must be initiated by constant expressions:
% \begin{lstlisting}[language=C++]
% constexpr int mf = 20; // 20 is a constant expression 
% constexpr int limit = mf + 1; // mf+1 is a constant expression 
% constexpr int sz = size(); // ok only if size is a constexpr function
% \end{lstlisting}

% Because a constant expression is one that can be evaluated at compile time, only literal types (arithmetic, reference, and pointer) types can be defined as constant expressions. Custom classes, library IO and \path{string} types are not literal types. We can point (or bind) to an object that remains at a fixed address.

% \path{constexpr} declaration applies to the pointer, not the type to which the pointer points:
% \begin{lstlisting}[language=C++]
% const int *p = nullptr; // pis a pointer to a constint
% constexpr int *q = nullptr; // qis a constpointer to int
% \end{lstlisting}
% \path{p} is a pointer to \path{const} (low-level), whereas \path{q} is a constant pointer (top-level).