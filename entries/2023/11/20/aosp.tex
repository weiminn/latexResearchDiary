% mainfile: ../../../../master.tex
\subsection{Rudimentary AOSP HAL Application}
\label{task:20231120_aosp}

Create Rudimentary Service source file, \path{rudi.cpp}, in \path{AOSP/device/generic/goldfish/app/wei_daemon}
\begin{lstlisting}[language=C++]
#include <unistd.h> 
#include <stdio.h> 
#include <android/log.h> 

#define DELAY_SECS 2 
#define ALOG(msg) __android_log_write(ANDROID_LOG_DEBUG, "WEIMINN PROJ", msg)

int main(int argc, char *argv[]) {
    ALOG("STARTING WEIMINN PROJECT");

    int n = 0; 
    while (true) {
        sleep(DELAY_SECS);
        n++; 

        ALOG("TESTING");
    }
}
\end{lstlisting}

And create Soong build file, \path{Android.bp} in the same folder:
\begin{lstlisting}[language=C++]
cc_binary { 
    name: "weiminn_daemon", 
    relative_install_path: "hw", 
    init_rc: ["init.weiminn.rc"], 
    header_libs: [
        "liblog_headers", 
    ], 
    srcs: [ 
        "weiminn.cpp" 
    ], 
    shared_libs: [ 
        "liblog", 
        "libcutils", 
    ], 
    static_libs: [ 
    ], 
    vendor: true, 
    proprietary: true, 
}
\end{lstlisting}

Add \path{wei_rudi_daemon} to the \path{PRODUCT_PACKAGES += \} attribute of \path{AOSP/device/generic/goldfish/vendor.mk}

Add the startup script below to the end of the \path{init.rc} file:
\begin{lstlisting}
service wei_rudi_daemon /vendor/bin/hw/wei_rudi_daemon
    class main
    user system
    group system
    oneshot
\end{lstlisting}

But the service cannot start due to SE Policy not being implemented for the service yet:
\begin{lstlisting}
11-21 02:13:54.242  1862  1862 W cp : type=1400 audit(0.0:231): avc: denied { getattr } for path="/vendor/bin/hw/wei_rudi_daemon" dev="dm-3" ino=110 scontext=u:r:shell:s0 tcontext=u:object_r:vendor_file:s0 tclass=file permissive=0
\end{lstlisting}


% \subsubsection*{Implementing the HAL}

% Device driver and its API are usually provided by a third-party hardware provider. Shim code include \path{.h} files for one or more device drivers.

% New devices can communicate with the processor via USB which is popular. Even without driver, the device can be accessed with generic USB commands. 
% \textit{libusb}\footnote{\url{https://libusb.info}} is a portable, user-mode, and USB-version agnostic library for using USB devices, and supports Android. If the device has a driver, it is likely to be a specialization of the USB command.

% \begin{lstlisting}[language=C++]
% #ifndef PROXIMITY_HAL_H 
% #define PROXIMITY_HAL_H 

% #include <hardware/hardware.h> 

% #define ACME_PROXIMITY_SENSOR_MODULE "libproximityhal" 

% typedef struct proximity_sensor_device proximity_sensor_device_t; 

% struct value_range { 
%     int min; int range; 
% }; 

% typedef struct proximity_params { 
%     struct value_range precision; 
%     struct value_range proximity; 
% } 

% proximity_params_t; 

% struct proximity_sensor_device { hw_device_t common; 
%     int fd; 
%     proximity_params_t params; 
%     int (*poll_sensor)(proximity_sensor_device_t *dev, int precision); 
% }; #endif // PROXIMITY_HAL_H

% \end{lstlisting}