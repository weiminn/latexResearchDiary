% mainfile: ../../../../master.tex
\subsection{Fixing SELinux Policy to Start System Service}
\label{task:20231121_aosp}


Add \path{seclabel} to the startup script:
\begin{lstlisting}[language=sh]
service weiminn_daemon /vendor/bin/hw/weiminn_daemon
    class main
    user system
    group system
    seclabel u:r:ueventd:s0
\end{lstlisting}
Add \texttt{start weiminn\_daemon} under \texttt{on early-init} right after \texttt{start ueventd}.

Got a new permission denied error this time:
\begin{lstlisting}
11-21 09:45:47.732 0 0 E init : cannot execv('/vendor/bin/hw/weiminn_daemon'). See the 'Debugging init' section of init's README.md for tips: Permission denied
11-21 09:45:47.733 0 0 I init : Service 'weiminn_daemon' (pid 1402) exited with status 127
11-21 09:45:47.733 0 0 I init : Sending signal 9 to service 'weiminn_daemon' (pid 1402) process group...    
\end{lstlisting}

Adding \path{device/generic/goldfish/sepolicy/x86/weiminn.te} with following contents:
\begin{lstlisting}
type weiminn, domain; 
permissive weiminn;
type weiminn_exec, vendor_file_type, exec_type, file_type; 

init_daemon_domain(weiminn)     
\end{lstlisting}
and changed \path{seclabel} of the startup script to \texttt{seclabel u:r:weiminn\_exec:s0}. Still got the same error.