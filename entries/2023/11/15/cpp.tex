% mainfile: ../../../../master.tex
\subsection{C++ Primer}
\label{task:20231115_cpp}

\subsubsection{Scopes of Names}

Most scopes in C++ are delimited by curly braces.
\begin{lstlisting}[language=C++]
#include <iostream> 
int main() { 
    int sum = 0; 
    // sum values from 1 through 10 inclusive 
    for (int val = 1; val <= 10; ++val) 
        sum += val; // equivalent to sum=sum+val 
    std::cout << "Sum of 1 to 10 inclusive is " 
        << sum << std::endl; 
    return 0; 
}
\end{lstlisting}
In above program, \texttt{main} - like most names defined outside a function - has \textbf{global scope} and thus, is accessible throughout the program. \texttt{sum} has \textbf{block scope} and is accessible from its point of declaration throughout the rest of the \texttt{main} function. \texttt{val} is defined in the scope of the \texttt{for} statement and can be used in that statement but not elsewhere in \texttt{main}.

Names declared in the outer scope can also be redefined in an inner scope although it is always a bad idea:
\begin{lstlisting}[language=C++]
#include <iostream> 
// Program for illustration purposes only: It is bad style for a function 
// to use a global variable and also define a local variable with the same name 
int reused = 42; // reused has global scope
int main() { 
    int unique = 0; // unique has block scope 
    
    // output #1: uses global reused;prints 42 0 
    std::cout << reused << " " << unique << std::endl; 
    
    int reused = 0; // new, local object named reused hides global reused 
    // output #2: uses local reused; prints 0 0 
    std::cout << reused << " " << unique << std::endl; 
    
    // output #3: explicitly requests the global reused; prints 42 0 
    std::cout << ::reused << " " << unique << std::endl; 
    
    return 0;
}
\end{lstlisting}
When the scope operator (\textbf{\texttt{::} operator}) has an empty LHS, it is a request to fetch the name on the RHS from the global scope.

\subsubsection{References}

A \textbf{reference} defines an alternative name for an object. A reference type can be defined by writing a declarator of the form \texttt{\&d} where \texttt{d} is the name being declared:
\begin{lstlisting}[language=C++]
int ival = 1024; 
int &refVal = ival; // refVal refers to (is another name for) ival 
int &refVal2; // error: a reference must be initialized
\end{lstlisting}
When we define a reference, instead of copying the initializer’s value, we bind the reference to its initializer. Once initialized, a reference remains bound to its initial object. There is no way to rebind a reference to refer to a different object. Because there is no way to rebind a reference, references must be initialized.

A reference is not an object. Instead, a reference is just another name for an already existing object. Thus, \textit{all} operation on that reference are actually operations on the object to which the reference is bound:
\begin{lstlisting}[language=C++]
refVal = 2; // assigns 2 to the object to which refVal refers, i.e., to ival 
int ii = refVal; // same as ii=ival
\end{lstlisting}

Becuase references are not objects, we may not define a reference to a reference.
We can define references in a single definition with each identifier that is a reference being preceded by the \texttt{\&} symbol:
\begin{lstlisting}[language=C++]
int i = 1024, i2 = 2048; // i and i2 are both ints 
int &r = i, r2 = i2; // r is a reference bound to i; r2is an int 
int i3 = 1024, &ri = i3; // i3 is an int; riis a reference bound to i3 
int &r3 = i3, &r4 = i2; // both r3and r4are references
\end{lstlisting}

\subsubsection{Pointers}
Like references, pointers are used for indirect access to other objects. Unlike a reference, a pointer is an object in its own right. Pointers can be assigned and copied; a single pointer can point to several different objects over its lifetime. Unlike a reference, a pointer need not be initialized at the time it is defined. Like other built-in types, pointers defined at block scope have undefined value if they are not initialized.

We define a pointer type by writing a declarator of the form *d,whered is the name being defined. The * must be repeated for each pointer variable:
\begin{lstlisting}[language=C++]
int *ip1, *ip2; // both ip1 and ip2 are pointers to int 
double dp, *dp2; // dp2 is a pointer to double; dp is a double
\end{lstlisting}

A pointer holds the address of another object. We get the address of an object by using the address-of operator (\textbf{\texttt{\&} operator}):
\begin{lstlisting}[language=C++]
int ival = 42; 
int *p = &ival; // p holds the address of ival; p is a pointer to ival

double dval; 
double *pd = &dval; // ok: initializer is the address of a double 
double *pd2 = pd; // ok: initializer is a pointer to double 
int *pi = pd; // error: types of pi and pd differ 
pi = &dval; // error: assigning the address of a doubletoapointertoint
\end{lstlisting}

We can use the dereference operator (\textbf{\texttt{*} operator}) to access that object:
\begin{lstlisting}[language=C++]
int ival = 42; 
int *p = &ival; // p holds the address of ival; p is a pointer to ival 
cout << *p; // * yields the object to which p points; prints 42

*p = 0; // * yields the object; we assign a new value to ival through p 
cout << *p; // prints 0
\end{lstlisting}
When we assign to \texttt{*p}, we are assigning to the object to which p points. We may dereference only a valid pointer that points to an object.

\texttt{void*} is a special pointer type that can hold the address of any object. Its useful for when the type of the object at that address is unknown:
\begin{lstlisting}[language=C++]
double obj = 3.14, *pd = &obj; // ok: void*can hold the address value of any data pointer type 
void *pv = &obj; // objcan be an object of any type 
pv = pd; // pvcan hold a pointer to any type
\end{lstlisting}

The modifiers \texttt{*} and \texttt{\&} do not apply to all variables defined in a single statement:
\begin{lstlisting}[language=C++]
int* p1, p2; // p1 is a pointer to int; p2is an int
int *p1, *p2; // both p1and p2are pointers to int
\end{lstlisting}

As pointers are objects in memory, they also have addresses of their own. Therefore, we can store the address of a pointer in another pointer:
\begin{lstlisting}[language=C++]
int ival = 1024; 
int *pi = &ival; // pi points to an int 
int **ppi = &pi; // ppi points to a pointer to an int
\end{lstlisting}
We indicate each pointer level by its own \texttt{*}. Dereferencing a pointer to a pointer yields the pointer. So in this case, you must dereference twice to access the underlying object.